% This is lnicst.tex the demonstration file of the LaTeX macro package for
% Lecture Notes of the Institute for Computer Sciences, Social-Informatics
% and Telecommunications Engineering series from Springer-Verlag.
% It serves as a template for authors as well.
% version 1.0 for LaTeX2e
%
\documentclass[lnicst]{svmultln}
%
\usepackage{graphicx}
\usepackage{makeidx,latexsym,amssymb,amsfonts,amsmath}  % allows for indexgeneration
\usepackage[utf8]{vietnam}
\usepackage{xcolor}
% \makeindex          % be prepared for an author index
%
\begin{document}
%
\begin{center}
\Large{\bf Model for Dominating Authors}
\end{center}


Theo thảo luận của anh Vũ và anh Hải về việc đưa trọng số đánh giá vai trò của tác giả bài báo tương ứng với vị trị của tác giả trong bài báo, Kiên có ý xây dựng mô hình như sau: (Ý tưởng này dựa trên ví dụ của anh Hải, và Kiên cũng đồng tình với ý tưởng này)
\\[-2mm]

Giả sử một publication $p$ có $k$ authors theo thứ tự: $A_1, A_2,\ldots, A_k$. Khi đó chúng ta xác định một hàm trọng số $\beta_1,\beta_2,\ldots,\beta_k$ (phụ thuộc theo $k$) để đánh giá mức ảnh hưởng của $p$ lên authors thõa mãn:
\[\beta_1 \geqslant \beta_2 \geqslant \cdots \geqslant \beta_k \qquad and \qquad \beta_1 + \beta_2 + \cdots + \beta_k = 1.\]

Khi đó
\[\qquad\qquad\qquad\qquad T(p,A_i) = W_{12}(p,A_i) = \beta_i \qquad\qquad \forall i = 1,\ldots,k.\]
\textcolor{red}{
Vấn đề chúng ta cần xác định $T(A_i,p) = W_{21}(A_i,p)$???}
\\[-2mm]

Xét một authors $A$ có $n$ bài báo $p_1,p_2,\ldots,p_n$. Dựa vào vị trí của $A$ trong $n$ bài báo này chúng ta xác định được ảnh hưởng của $n$ bài báo này đến $A$ như sau:
\[\qquad\qquad\qquad\qquad T(p_j,A) = \alpha_j \qquad\qquad \forall j = 1,\ldots,n.\]

Khi đó chúng ta không nên xem $A$ có $n$ bài báo mà nên xem $A$ có
\[\alpha_1 + \alpha_2 + \cdots + \alpha_n\]
bài báo. Từ đó ảnh hưởng của author $A$ đến các bài báo được xác định như sau:
\[\qquad\qquad T(A,p_j) = W_{21}(A,p_j) = \frac{\alpha_j}{\alpha_1 + \alpha_2 + \cdots + \alpha_n} \qquad\forall j = 1,\ldots,n.\]

Chúng ta có thể xác định các trọng số $\beta_1,\beta_2,\ldots,\beta_k$ như ý tưởng anh Hải đưa ra.
\\[-2mm]

Em viết sơ công thức tính để anh Hải cho chương trình chạy. Sau khi có kết quả, nếu tốt thì em sẽ viết công thức lại nghiêm túc hơn để chúng ta đưa vào bài báo. Anh Vũ và anh Hải xem rồi cho ý kiến nhé.
\\[-2mm]

Hải comment:
Anh đồng ý với Kiên về tổng các $\alpha_i$, nhưng làm thế nào để so sánh và đánh giá các giá trị $\alpha_1$,$\alpha_2$...?
Có cách nào tính được từ các $\beta_j$ bằng việc chuyển vị như ví dụ của anh nêu ra không?
Em Kiên.
\\[-2mm]

Kiên comment:
Các giá trị $\alpha_1,\alpha_2,\ldots,\alpha_n$ chính là các giá trị $\beta$ đấy anh. Em viết như thế này để anh code chương trình dễ hơn, không thông qua ma trận. Em ví dụ nhé: Giả sử author $A$ có 5 publications $p_1, p_2, p_3, p_4, p_5$.
\begin{itemize}
\item $A$ đứng vị trí 1 trong publication $p_1$ và publication $p_1$ có 3 authors. Suy ra $A$ có 1/2 ($\alpha_1$) publication đối với $p_1$ 
\begin{center}
(chúng ta xác định theo trọng số 1/2, 1/3 và 1/6)
\end{center}
\item $A$ đứng vị trí 2 trong publication $p_2$ và publication $p_2$ có 2 authors. Tương tự $A$ sẽ có 1/3 ($\alpha_2$) publication đối với $p_2$ 
\begin{center}
(chúng ta xác định theo trọng số 2/3 và 1/3)
\end{center}
\item $A$ đứng vị trí 1 trong $p_3$ và $p_3$ có 1 author. $A$ sẽ có 1 ($\alpha_3$) publication đối với $p_3$.
\item $A$ đứng vị trí 1 trong $p_4$ và $p_4$ có 5 authors. $A$ sẽ có 5/15 = 1/3 ($\alpha_4$) publication đối với $p_4$ 
\begin{center}
(chúng ta xác định theo trọng số 5/15, 4/15, 3/15, 2/15 và 1/15)
\end{center}
\item $A$ đứng vị trí 3 trong $p_5$ và $p_5$ có 6 authors. $A$ sẽ có 4/21 ($\alpha_5$) publication đối với $p_5$ 
\begin{center}
(chúng ta xác định theo trọng số 6/21, 5/21, 4/21, 3/21, 2/21 và 1/21)
\end{center}
\end{itemize}

Như vậy đối với 5 publications, theo cách tính chúng ta $A$ sẽ có số bài báo là:
\[\alpha_1 + \alpha_2 + \ldots + \alpha_5 = \frac{1}{2} + \frac{1}{3} + 1 + \frac{1}{3} + \frac{4}{21} = \frac{33}{14}\]

Vậy điểm ảnh hưởng của $A$ lên các publications và publications lên $A$ sẽ là:
\[T(p_1,A) = \alpha_1 = \frac{1}{2} \qquad and \qquad T(A,p_1) = \frac{\alpha_1}{33/14} = \frac{7}{33}\]
\[T(p_2,A) = \alpha_2 = \frac{1}{3} \qquad and \qquad T(A,p_2) = \frac{\alpha_2}{33/14} = \frac{14}{99}\]
\[T(p_3,A) = \alpha_3 = 1 \qquad and \qquad T(A,p_3) = \frac{\alpha_3}{33/14} = \frac{14}{33}\]
\[T(p_4,A) = \alpha_4 = \frac{1}{3} \qquad and \qquad T(A,p_4) = \frac{\alpha_4}{33/14} = \frac{14}{99}\]
\[T(p_5,A) = \alpha_5 = \frac{4}{21} \qquad and \qquad T(A,p_5) = \frac{\alpha_5}{33/14} = \frac{4\times14}{33\times21}\]

Anh Hải xem có đúng ý anh chưa nhé.
%

\textcolor{red}{
Hi Kiên, như comment trong email: model này đã làm cho 
\[W_{12}\ * W_{21} = I\] }

Do đó, anh đề nghị cách tính $W_{21}$ như sau:

Xét một authors $A$ có $n$ bài báo $p_1,p_2,\ldots,p_n$. Dựa vào \textcolor{red}{thứ tự thời gian xuất bản và  vị trí của $A$ trong $n$ bài báo này chúng ta xác định được ảnh hưởng của $n$ bài báo này đến $A$ theo quan điểm nếu bài báo được viết trước thì tác giả bỏ công sức bỏ ra nhiều hơn bài sau hay có ảnh hưởng lớn hơn}, và như thế chúng ta có:
\[\qquad\qquad\qquad\qquad T(p_j,A) = \alpha_j *  t_k \qquad\qquad \forall j = 1,\ldots,n. \forall k = 1,\ldots,n. \]

Khi đó chúng ta không nên xem $A$ có $n$ bài báo mà nên xem $A$ có
\[\alpha_1*t_1 + \alpha_2*t_2 + \cdots + \alpha_n*t_n\]
bài báo. Từ đó ảnh hưởng của author $A$ đến các bài báo được xác định như sau:
\[\qquad\qquad T(A,p_j) = W_{21}(A,p_j) = \frac{\alpha_j*t_j}{\alpha_1*t_1 + \alpha_2*t_2 + \cdots + \alpha_n*t_n} \qquad\forall j = 1,\ldots,n.\]

Các $\alpha_j$ xác định từ $\beta_j$ như phần diễn giải của Kiên ở trên và $t_k$ xác định theo việc đánh chỉ số năm xuất bản bài báo.
Ví dụ : Giả sử author $A$ có 5 publications $p_1, p_2, p_3, p_4, p_5$ với số năm xuất bản là $2011,2008,2003, 2009,2008$ thì các trọng số $t_k$ tương ứng là $1/15,3.5/15,5/15,2/15,3.5/15$)

Anh nghĩ thêm yếu tố thứ tự thời gian vào như thế này thì có gì đó chưa được hợp lý cho lắm nhưng chưa nghĩ ra cách nào hay hơn. Nhưng tạm thời sẽ tránh được việc tích nhân $W_12*W_21=I$ 
Và công thức cũng đã phức tạp lên nhiều rồi 
Rõ ràng nếu có 01 tác giả có xuất bản các bài báo cùng thời điểm thì công thức trở cũng chỉ về đơn vi đối với tác giả này mà thôi.

Nhờ Kiên và anh Vũ xem và cho thêm ý kiến nhé.

\end{document}

